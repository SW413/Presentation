\section{Implementering og demonstration}
\begin{frame}{Implementering og demonstration}
	\framesubtitle{Indhold}
    \begin{itemize}
    		\item \textbf{Implementering}
    		\begin{itemize}
		\item Model - Repræsentation af data
		\item Controller - Relevant logik
		\item View - Det brugeren ser
        \end{itemize}
        \vspace{12pt}
        \item \textbf{Demonstration} 
        \begin{itemize}
        \item Typisk workflow        
        \item Mobil udgave  
        \end{itemize}        
    \end{itemize}
\end{frame}

%% Model
\subsection{Model}
\begin{frame}{Model}
	\framesubtitle{Indkøbslister}
	\texttt{ShoppingList.cs}
	\hbox{\hspace{5 mm}\scalebox{0.8}{
	\lstinputlisting{code/model/ShoppingList.cs} }}	
\end{frame}
\begin{frame}{Model}
	\framesubtitle{Varer}	
	\texttt{Item.cs}
	\hbox{\hspace{5 mm}\scalebox{0.8}{
	\lstinputlisting{code/model/Item.cs}	}}
\end{frame}

%% Controller
\subsection{Controller}
\begin{frame}{Controller}
	\begin{itemize}
		\item CRUD - \texttt{Create()}, \texttt{Index()} og \texttt{Details()}, \texttt{Edit()}, \texttt{Delete()}
		\item \texttt{DataBaseContext} giver adgang til model-laget
		\vspace{15pt}
		\item Tilføje og fjerne varer, tilbud og ingredienser
		\item Kommunikation med API
		\item Information til brugere
	\end{itemize}
	%Indsæt relevant kode måske? eller tab til VS 
\end{frame}

%% View
\subsection{View}
\begin{frame}{View}
	\begin{itemize}
		\item Overblik - \texttt{Index.cshtml}
		\begin{itemize}
			\item \texttt{foreach} der ittererer over alle objekterne fx indkøbslister
			\item F.eks. indkøbslister, tilbud og opskrifter
		\end{itemize}
		\item Specifik objekter - \texttt{Details.cshtml}
		\begin{itemize}
			\item \texttt{foreach} der ittererer over eventuelle lister, som objektet har fx varer på en indkøbsliste
			\item F.eks. en indkøbsliste eller opskrift
		\end{itemize}
	\end{itemize}
\end{frame}

\subsection{Demonstration}
\begin{frame}{Demonstration}
	\framesubtitle{Typisk workflow}
	\begin{itemize}
	\item Oprette bruger
	\item Bruge indkøbsliste og tilbud
	\item Finde opskrift og lave min egen
	\item Ændre mine præferencer
	\end{itemize}
	\href{http://james:8080}{Demonstration af ProjectFood}
\end{frame}